\documentclass{article}
\setlength{\parindent}{0pt}
\setlength{\parskip}{1em}
\usepackage[a4paper, total={6in, 9in}]{geometry}
\usepackage{soul}
\usepackage{enumerate}
\usepackage{xcolor,url}

\newcommand{\rcomment}[1]{\vspace{0.3cm} \item \textbf{Reviewer's Comment:} {\em #1}}
\newcommand{\tcomment}[1]{\vspace{0.3cm} {\color{red} \item \textbf{Reviewer's Comment:} {\em #1}}}
\newcommand{\jcomment}[1]{\vspace{0.3cm} {\color{blue} \item \textbf{Reviewer's Comment:} {\em #1}}}

\newcommand{\response}{\vspace{0.2cm} \textbf{Response: }}
\newcommand{\jresponse}[1]{\vspace{0.2cm} \textbf{Response: }{\color{blue} {\em #1}}}

\title{{\normalsize Revision Report for the Computer Journal submission}\\
{\Large\em ``A Heuristic Approach towards Drawings of Graphs with\\High Crossing Resolution''}}
\author{}
\date{}

%\author{Patrizio Angelini, Michael A. Bekos, Giuseppe Liotta, Fabrizio Montecchiani}

\author{M.~Bekos, H.~F\"orster, C.~Geckeler, L.~Holl\"ander, M.~Kaufmann, A.~Spallek, J.~Splett}

\begin{document}

\maketitle

\noindent We thank all reviewers for their very helpful comments and suggestions. We have addressed all the raised issues. Details are provided below, where we list all reviews together with a detailed response to each suggestion by the reviewers.

\newpage


% ==================================================================================
\section*{Reviewer 1}
% ==================================================================================

The authors present a heuristic that aims to maximize the crossing resolution of graph. Their heuristic can be adopted to also aim towards maximizing angular and total resolution. The authors present an extensive experimental analysis which demonstrates that their heuristic outperforms the previously known approaches ([21,22], which are both ``spring-embedder'' like heuristics). The authors also demonstrate that their approach is efficient, resulting results in reasonable amounts of time. On the practical side they also produce grid drawings of restricted size, which is something previously known algorithms did not address. 

The paper is well written. The heuristic is presented in a clear and simple way so that it is possible to be programmed by other researchers/developers. Sufficient details are also provided on how to run each step of the heuristic efficiently. Given the limited applied previous work on the topic (most of the previous work focuses on the case of RAC graphs), i believe that their contribution is significant and it is definitely worth publishing. I suggest ACCEPTANCE subject to addressing the following 

MINOR issues:

\begin{itemize}
\rcomment{pg 2, ln 19, right. Remove ``Unless otherwise specified''. No directed or non-simple graphs are considered.}

\response{Done.}

\rcomment{pg 3, ln 30 right. change to ''... performed miserably''}

\response{Done.}

\rcomment{Table 1: Indicate in the caption or in the column headings that the last 2 columns correspond to the results presented in this paper.}

\response{Done.}

\rcomment{Table 2: Use a better heading than "our best" (for example, "no-time restriction"?)}

\response{Revised as suggested.}

\rcomment{pg 4, ln 41, left: Mention Section 4}

\response{Done.}

\rcomment{pg 4, ln 60, left: The text followin "In other words.." is not implied by the formula. The formula should be: $c(\Gamma_{k}) \geq c(\Gamma_{k-1}), ~~0 \leq k \leq i-1$ or similar.}

\response{Done.}

\rcomment{pg 4, ln 46, right: ``or of a vertex that is in the neighborhood of the critical vertex...'' Why is this true in general? The edges incident to vertices in the neighborhood are not related to the crossing considered.}

\response{This is not easy to answer precisely. Indeed, the vertices in the neighborhood are not related to the crossing resolution of the current drawing. However, by also considering these vertices (and in particular by moving them), we have observed that in several cases it is possible to gain the necessary space for the actual critical vertices, i.e., those that define the drawing's crossing resolution, to move and thus to improve the drawing's crossing resolution. This is the idea underneath.}

\rcomment{pg 4, ln 52, right: If possible, provide an intuition why you are using graph-distance instead of geometric-distance.}

\response{Of course, one could naturally use the Euclidean distance as an alternative criterion for selecting the vertices to be moved. Here, we have chosen the graph-distance instead, mainly for reasons of efficiency. In the revised version, we have added a corresponding note discussing this observation.}

\rcomment{pg 5, ln 17, left: ``chooses uniformly at random''. This is not accurate. You are not using the uniform distribution. In the previous page you assigned probalilities to the vertices wrt their graph-distance from the critical vertex. Are you applying this rule. Please clarify.}

\response{Indeed, this statement was not accurate. The quoted text has been removed in the revised version.}

\rcomment{pg 5. ln 30 left: Mention that $\rho$ is also a parameter of the algorithm.}

\response{In the revised version, we mention that $\rho$ is a parameter of the algorithm as follows: ``where $\rho > 0$ is an integer parameter of the algorithm''.}

\rcomment{General comment for section 2. Mention early in the section that your heuristic, as described, produces real coordinates and that you explain in Sect 2.3 how to produce a grid drawing. (Having seen Table 2, a reader expects to get an algorithm that produces grid drawing at the first place).}

\response{Revised according to the reviewer's suggestion.}

\rcomment{General comment for section 2. Have a discussion regarding the real-number computations involved. How many decimal points used for crossing resolution, vertex position, etc.}

\response{We have made a corresponding note about this well-known issue in Section~2; in particular, at the point in which we describe how to compute the set of critical vertices. Note that in our approach we do not adopt any special data structure from Computational Geometry to increase the precision in our computations. Hence, the precision depends on the used machine.}

\rcomment{pg 6, ln 11, right: "Aspect ratio" This is not the traditional definition. Aspect ratio is typically defined as the ratio of largest to smaller side-length of the enclosing renctange of the drawing. The smaller edge to largest edge ratio is more of a
"resolution" type measurement.} 

\response{Indeed, we were misusing the term ``aspect ratio''. The correct term is ``edge-length ratio''. In the revised version, this problem has been resolved. We have also added a reference to a recent paper introducing and studying this measure for outerplanar graphs (Sylvain Lazard, William J. Lenhart, Giuseppe Liotta: On the edge-length ratio of outerplanar graphs. Theor. Comput. Sci. 770: 88-94 (2019)).}

\rcomment{Figures 4,5,7,8. In caption (a) you state that it is the crossing resolution. In these figures it is either angular of total resolution.}

\response{We have resolved this problem.}

\rcomment{Figure 9. Clearly state that the double line refers to an algorithm that is unrestricted wrt the grid size. (So far, you have used the term "unrestricted" wrt the time devoted to each run; the user then expectes to have 4 such lines, one per grid size).}

\response{In the revised version, we have revised the caption of Figure~9 to make this clear.}

\rcomment{pg 16, line 38, left: "naturalLY observe".}

\response{Done.}

\tcomment{General comment: It should be also interesting to compare the crossing resolution between the "non-grid" and the "grid" versions of your algorithm}

\response{\textcolor{red}{Which version of the algorithm do we use in the experiments?}}
\end{itemize}


% ==================================================================================
\newpage
\section*{Reviewer 2}
% ==================================================================================

The paper studies how to compute straight-line drawings of graphs such that the angle at which two edges cross is as large as possible. Deciding whether a graph has a drawing where each pair of crossing edges cross at a 90-degree angle is an NP-hard problem. The literature contains three heuristic approaches that aim to optimize the crossing angle, two force-directed approaches (Argyriou et al, Huang et al.) and a randomized approach (Demel et al., aka CoffeeVM). The randomized approach has been published at the same conference as the preliminary version of the paper at hand.  The authors participated with their approach at the annual Graph Drawing Contest 2018. Their approach outperformed the winner (CoffeeVM) of the 2018 edition of the contest [0,1].


The paper introduces a heuristic based on local modifications. The approach iteratively moves a vertex v to a better position. Since it is unclear how to efficiently compute a good position of a vertex, the authors follow a randomized approach. They compute a random sample S following certain restrictions and move to v to the position in S that maximizes the crossing angle of the edges incident to v. In order to escape local maxima, the authors introduce some parameters to control the sample space and the vertex candidates. The approach is evaluated in an experimental evaluation. First, the authors show that the approach computes larger crossing angles compared to the force-directed approaches. Moreover, they provide insights on the interaction between large crossing angle affects other metrics, in particular, the number of crossing and the aspect ratio of the drawing.

In contrast to the conference version, the full journal contains a significant amount of new material. Nevertheless, before publishing, I would like to see the following two concerns addressed.

\begin{itemize}
\tcomment{First, the authors claim (Page 2, L20) to be aware of only two approaches (Argyriou et al. and Huang et al.) that optimize the crossing angle. On the other hand, the introduction contains a comparison to a third approach (CoffeeVM). The abstract in [1] suggests that CoffeeVM follows a similar local randomized approach. In my opinion, it is only fair to the authors of CoffeeVM to discuss similarities and differences between the two approaches.  Moreover, the vertex movement paradigm is an established technique and has successfully applied to graph theoretic and geometric problems. I propose to add [1-6] as relevant related work.}

\response{}

\tcomment{Second, I think that the structure of the evaluation can be improved by formulating a set of research questions. For example:

I) Does algorithm A compute larger crossing angles than algorithm B?

II) What is the angular resolution in drawings with small crossing angles?

III) How many iterations need the algorithms to converge?

Once the research questions are stated, one can think about proper setup and tools to investigate these questions. Of course with an experimental evaluation, it is unlikely that we are able to verify the research questions in their entirety. But we can collect indications that the statements are likely to be true (on specific data sets). One indicator that can be compared is the mean of two datasets, that is is the sole statistic used by the authors. }

\response{}

\tcomment{In my opinion, the mean should be accompanied by more statistics (at least the standard deviation), since the mean is sensitive to outliers. Moreover, I have the impression that the writing of the experimental evaluation lacks precision. A more formal style of writing reduces the possibility that statements can be misunderstood.
For example, the word /significant/ has a defined meaning, it implies that a statement is very unlikely and this has been confirmed with a proper statistical test. But comparing the means of two sequences of observations A and B does not imply significance. Instead of writing "A has significantly smaller crossing angles than B", I propose to write "The mean of A is (considerably) smaller than the mean of B", or "the mean of A is X degrees smaller than the mean of B". In the following, I give an incomplete list of statements that should be revised.}

\response{}

\tcomment{p7,c2, ~l42) 'It is immediate to see that our algorithm outperforms all other ones in terms of the crossing resolution[...]'.

The statement suggests an irrevocable truth but the only conclusion that you can draw from the plots is the following 'the mean crossing resolution of drawings obtained by your algorithm is considerably larger than the mean crossing resolutions of the remaining algorithms'. Alternatively, (globally) define the meaning of 'outperforms' "A outperforms B if the mean of A is X degrees smaller than the mean of B"}

\response{}

\tcomment{p7, c2, l54: While our unrestricted algorithm produces drawings with better crossing resolution no need to say /better/, it is /larger/}

\response{}

\tcomment{p8, c1, l56: ... has more or less comparable performance.... what does 'more or less' mean? Does the mean crossing resolution differ by at most x degrees?}

\response{}

\tcomment{p8, c1/2: ...which at the same time is significantly smaller;}

\response{}

\tcomment{p9, c1, l55: ... significant... I do not see why you are able to claim significance.}

\response{}

\tcomment{p10 c1, l47: For the vast majority of the graphs in the experiments; I am not sure whether you are still comparing the mean or the number of graphs that have a smaller angle. In the latter case there is no need to be vague: e.g. 'On 90\% of the graphs our algorithms compute drawings with a smaller crossing than drawings obtained by B'}

\response{}


\tcomment{p10, c1, l57: ...tend to produce drawings of the same total resolution on larger graphs 

What does 'tend to the same' mean? Either they are equal or they are not. 

How many vertices do the larger graphs have?    

I guess what you would like to say is along the following lines:

The mean values for the total resolution of drawings obtained by our algorithm and by Argyriou et al. differ by a most 5(?) degrees for graphs with more than 90 vertices.}

\response{}

\tcomment{p11, c2, ~l51: ...small graphs.... medium-size graphs

As before, it is not clear when a graph is small, medium-sized or large. In case that you would like to use these terms frequently, I propose to introduce graphs classes: S, M, L that partition the rome graphs based on their number of vertices.}

\response{}

\tcomment{p12, c1, l48: the angular resolution dominates the crossing resolution

I have problems understanding this statement. What does dominate mean?  Do you want to investigate the number of graphs whose crossing resolution is smaller than the total resolution? If so, I do not see how you can observe this based on the plots (even ignoring that my statement is on the number of graphs). Moreover, it seems strange to refactor this information from your plots when you could simply get this information from your data.}

\response{}
        
\end{itemize} 

\textbf{References:}

[0] Devanny W., Kindermann P., Löffler M., Rutter I. (2018) Graph Drawing Contest Report. In: Frati F., Ma KL. (eds) Graph Drawing and Network Visualization. GD 2017. Lecture Notes in Computer Science, vol 10692. Springer, Cham

[1] Demel A., Dürrschnabel D., Mchedlidze T., Radermacher M., Wulf L. (2018) A Greedy Heuristic for Crossing-Angle Maximization. In: Biedl T., Kerren A. (eds) Graph Drawing and Network Visualization. GD 2018. Lecture Notes in Computer Science, vol 11282. Springer, Cham

[2] Johan Ugander and Lars Backstrom. 2013. Balanced label propagation for partitioning massive graphs. In Proceedings of the sixth ACM international conference on Web search and data mining (WSDM '13). ACM, New York, NY, USA, 507-516. DOI: https://doi.org/10.1145/2433396.2433461 

[3] C. M. Fiduccia and R. M. Mattheyses, "A Linear-Time Heuristic for Improving Network Partitions," 19th Design Automation Conference, Las Vegas, NV, USA, 1982, pp. 175-181.
doi: 10.1109/DAC.1982.1585498

[4] Kernighan, B. W. and Lin, S. (1970), An Efficient Heuristic Procedure for Partitioning Graphs. Bell System Technical Journal, 49: 291-307. doi:10.1002/j.1538-7305.1970.tb01770.x

[5]  A Geometric Heuristic for Rectilinear Crossing Minimization
Marcel Radermacher, Klara Reichard, Ignaz Rutter, and Dorothea Wagner
2018 Proceedings of the Twentieth Workshop on Algorithm Engineering and Experiments (ALENEX). 2018, 129-138 

[6] Bläsius T., Radermacher M., Rutter I. (2017) How to Draw a Planarization. In: Steffen B., Baier C., van den Brand M., Eder J., Hinchey M., Margaria T. (eds) SOFSEM 2017: Theory and Practice of Computer Science. SOFSEM 2017. Lecture Notes in Computer Science, vol 10139. Springer, Cham

\end{document}
