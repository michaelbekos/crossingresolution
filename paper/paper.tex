\documentclass{llncs}
\usepackage{amsmath,amssymb}
\usepackage{graphicx,rotating}
\usepackage[font=small]{subfig,caption}
\usepackage{algorithm2e}
\usepackage{todonotes,lineno,paralist,soul}
\usepackage{placeins,rotating,hyperref}
\graphicspath{{figures/}}

\author{Michael~A.~Bekos, Henry~F\"orster, Christian-Marius~Geckeler, Lukas Holl\"ander, Michael~Kaufmann, Amad\"aus~Spallek, Jan~Splett}
\title{A Heuristic Approach towards Drawings of Graphs with High Crossing Resolution}

\institute{
Wilhelm-Schickhard-Institut f\"ur Informatik, Universit\"at T\"ubingen, Germany\\
\texttt{\{bekos,foersth,mk\}@informatik.uni-tuebingen.de}\\
\texttt{\{christian-marius.geckeler,jan-lukas.hollaender,amadaeus.spallek,jan.splett\} @student.uni-tuebingen.de}
}

% ============================================================
\begin{document}
\maketitle
% ==================================================================

\begin{abstract}
The \emph{crossing resolution} of a non-planar drawing of a graph is the minimum angle formed at the crossing points of the edges. Recent experiments have shown that the larger the crossing resolution is, the easier it is to read and interpret a drawing of a graph. However, maximizing the crossing resolution turns out to be an NP-hard problem in general and only heuristic algorithms are known that are mainly based on appropriately adjusting spring embedding algorithms. 
 
In this paper, we propose a new --randomization-based-- heuristic algorithm for the crossing resolution maximization problem and we experimentally compare it against the known ones from the literature. Our experimental evaluation indicates that the new heuristic produces drawings with significantly better crossing resolution, but this comes at the cost of slightly higher running time, especially when the input graph is large and dense. 
\end{abstract}

\section{Introduction}
\label{sec:introduction}

In Graph Drawing, there exist a really reach literature and a wide range of techniques for drawing planar graphs; see, e.g.,~\cite{DBLP:journals/combinatorica/FraysseixPP90,DBLP:conf/gd/GutwengerM98,DBLP:journals/algorithmica/Kant96}. However, drawing a non-planar graph, and in particular when it does not have some special structure (e.g., degree restricted), is a difficult and challenging task, mainly due to the edge crossings which negatively affect the drawing's quality~\cite{DBLP:journals/iwc/Purchase00}. As a result, the proposed techniques are significantly fewer (e.g., crossing minimization heuristics~\cite{DBLP:journals/algorithmica/EadesW94,DBLP:journals/tsmc/SugiyamaTT81}, energy-based layout algorithms~\cite{DBLP:journals/congnum/Eades84,DBLP:journals/spe/FruchtermanR91}); for an overview refer to~\cite{DBLP:books/ph/BattistaETT99,DBLP:conf/dagstuhl/1999dg,DBLP:reference/crc/2013gd}.

In this context, Huang et al.~\cite{DBLP:conf/apvis/Huang07,DBLP:journals/vlc/HuangEH14} a decade ago introduced some important statistical evidence (through a series of eye-tracking experiments), that edge crossings may not negatively affect the drawing's quality too much (and therefore the human's ability to read and interpret it), when the angles formed by the crossing edges are large. In other words, while prior to these experiments it was commonly accepted that mainly the number of crossings is the most important parameter for judging the quality and readability of a non-planar graph drawing, it turned out that the types of edge crossings also matter. As a result, a new and prominent research direction was initiated, which is unofficially recognized under the term ``beyond planarity''~\cite{Shonan2016,Dagstuhl2016,SoCG2017} and mainly focuses on graphs and their properties, when different constraints on the types of edges crossings are imposed; refer to~\cite{DBLP:journals/corr/abs-1804-07257} for a recent survey. 

Formally, the minimum angle formed by any two crossed edges in a drawing is refer to as its \emph{crossing resolution}. Analogously, the crossing resolution of a graph is defined as the maximum crossing resolution over all its drawings. Clearly, the crossing resolution of a non-planar graph is at most $90^\circ$, while a graph that admits a drawing with crossing resolution $90^\circ$ is refer to as \emph{right-angle-crossing} graph or \emph{RAC} graph, for short. For these graphs, several results, mostly of theoretical nature, are known (refer to Section~\ref{sec:relatedwork} for a short overview). Notably, RAC graph are sparse (they contain at most $4n-10$ edges~\cite{DBLP:journals/tcs/DidimoEL11}, where $n$ denotes the number of  vertices), while deciding whether a graph is RAC is NP-hard~\cite{DBLP:journals/jgaa/ArgyriouBS12}. 

The latter result is already an indication that the crossing resolution maximization problem might also be difficult, even though, formally, its complexity has not been settled yet for values of the crossing resolution smaller than $90^\circ$. Also, the literature is significantly more limited, when restricting the crossing resolution to be smaller than $90^\circ$, as also evidenced by Section~\ref{sec:relatedwork}. 

From a practical point of view, we are only aware of two methods that aim in drawings with high crossing resolution; both of them are adjustments of spring embedding algorithms~\cite{DBLP:journals/congnum/Eades84}. The first one is due to Huang, Eades, Hong and Lin~\cite{DBLP:journals/vlc/HuangEHL13}, while the second one due to Argyriou, Bekos and Symvonis~\cite{DBLP:journals/cj/ArgyriouBS13}. Common in both algorithms are appropriate forces, which are applied on the endvertices of every pair of crossing edges. Each of the two algorithms uses a different way to compute (the direction and the magnitude of) the forces, but the underlying idea in both algorithms is the same: the smaller the corresponding crossing angles are, the larger are the magnitudes of the forces applied at their endvertices. 

In this paper, we attack the crossing resolution maximization problem by following a completely different approach. We suggest a simple and quite intuitive randomization method for computing drawings with high crossing resolution, which in practise turned out to be quite efficient\footnote{Note that, in general, randomization is a technique that has not been deeply examined in Graph Drawing, as it seems to be difficult to computed the expected quality of the produced drawings. A notable exception is the randomized approach by Goldschmidt and Takvorian~\cite{DBLP:journals/networks/GoldschmidtT94} for computing large planar subgraphs.}. However, since we could not provide any theoretical guarantee on the expected quality of the produced drawings (mainly due to the nature of the problem itself), we decided to follow a more practical approach. To this end, we implemented our algorithm and the ones presented in~\cite{DBLP:journals/vlc/HuangEHL13} and~\cite{DBLP:journals/cj/ArgyriouBS13}, and we experimentally compared the crossing resolutions of the drawings produced by them for standard benchmark graphs (such as the Rome and the North graphs) that are widely used in Graph Drawing for comparing different drawing algorithms. Our experimental evaluation indicates that our new method significantly outperforms the ones that are based on adjustments of spring embedding algorithms~\cite{DBLP:journals/vlc/HuangEHL13,DBLP:journals/cj/ArgyriouBS13} in terms of crossing resolution, but this comes at the cost of slightly higher running time, especially for large and dense graphs. 

The remainder of this paper is structured as follows. Section~\ref{sec:relatedwork} overviews related works. In Section~\ref{sec:preliminaries}, we introduce preliminary notions and notation used throughout the paper. Our algorithm is presented in detail in Section~\ref{sec:algorithm} and is experimentally evaluated against the ones of Huang et al.~\cite{DBLP:journals/vlc/HuangEHL13} and Argyriou at al.~\cite{DBLP:journals/cj/ArgyriouBS13} in Section~\ref{sec:experiments}. We conclude in Section~\ref{sec:conclusions} with interesting open problems and future~directions.
 
\section{Related Work}
\label{sec:relatedwork}

As already mentioned, the study of the crossing resolution maximization problem has mainly focused on (propertied of) RAC graphs, that is, on the optimal case of the crossing resolution maximization problem. The study was initiated by Didimo et al.~\cite{DBLP:journals/tcs/DidimoEL11}, who showed that an $n$-vertex RAC graph has at most $4n-10$ edges. 
Angelini et al.~\cite{DBLP:journals/jgaa/AngeliniCDFBKS11} presented results for two interesting variants, in which the input graph is either an acyclic digraph or a graph of bounded vertex degree. Argyriou et al.~\cite{DBLP:journals/jgaa/ArgyriouBS12} showed that the problem of deciding whether a graph is RAC is NP-hard. Didimo et al.~\cite{DBLP:journals/ipl/DidimoEL10} completely characterized the complete bipartite RAC graphs. Di~Giacomo et al.~\cite{DBLP:journals/algorithmica/GiacomoDEL14}, and Hong and Nagamochi~\cite{DBLP:conf/wg/HongN15} studied variants, in which the vertices of the input graph are restricted on two parallel lines and on the circumference of a circle, respectively. Eades and Liotta~\cite{DBLP:journals/dam/EadesL13} showed that the optimal RAC graphs (i.e., those of maximum density) are 1-planar (i.e., they admit drawings in which each edge is crossed at most once). Deciding, however, whether a 1-planar graph is RAC is NP-hard~\cite{DBLP:journals/tcs/BekosDLMM17}. Other relationships between the class of RAC graphs and subclasses of 1-planar graphs have also been studied by Bachmaier et al.~\cite{DBLP:journals/dam/BachmaierBHNR17} and Brandenburg et al.~\cite{DBLP:journals/tcs/BrandenburgDEKL16}. Note that the problem of finding RAC drawings has been also studied in the presence of bends; see e.g.~\cite{DBLP:journals/jgaa/AngeliniCDFBKS11,DBLP:journals/comgeo/ArikushiFKMT12,DBLP:journals/tcs/DidimoEL11,DBLP:journals/mst/GiacomoDLM11}. 

To the best of our knowledge, there is only one work, by Dujmovic et al.~\cite{DBLP:journals/cjtcs/DujmovicGMW11}, which studies the crossing resolution maximization problem by relaxing the right-angle constraint on the crossing angles of the computed drawings. More precisely, in their work, Dujmovic et al.~\cite{DBLP:journals/cjtcs/DujmovicGMW11} proved that an $n$-vertex graph, whose crossing resolution is at least $\alpha$ has at most $(3n-6)\pi/\alpha$ edges. Corresponding density results for the case, in which few bends are allowed along each edge, are known by Ackerman et al.~\cite{DBLP:journals/siamdm/AckermanFT12} and Di Giacomo et al.~\cite{DBLP:journals/mst/GiacomoDLM11}.  

An immediate observation emerging from the above literature overview is that the focus of the study of the crossing resolution maximization problem has been primarily on theoretical aspects of the problem (as naturally expected, of course). The absence of approaches for computing drawings with high crossing resolution in practice is rather obvious and it can be justified, up to a certain point, by fact that the problem turns to be NP-hard, when asking for drawings with optimal crossing resolution. Exceptions are the works of Huang et al.~\cite{DBLP:journals/vlc/HuangEHL13} and Argyriou at al.~\cite{DBLP:journals/cj/ArgyriouBS13}, which are both based on the spring-embedding technique~\cite{DBLP:journals/congnum/Eades84}. According to this technique, a graph is modelled as a physical system with forces acting on it, and a (good) drawing is obtained by an equilibrium state of the system; for an introduction and a detailed discussion of several different variants refer to~\cite{DBLP:books/ph/BattistaETT99}. 

\section{Preliminaries}
\label{sec:preliminaries}

Unless otherwise specified, in this paper we consider simple undirected graphs. Let $G=(V,E)$ be such a graph. The degree of vertex $u\in V$ of $G$ is denoted by $d(u)$. The degree $d(G)$ of  graph $G$ is defined as the maximum degree of its vertices, i.e., $d(G)=\max_{u\in V}d(u)$.

Given a drawing $\Gamma(G)$ of $G$ on the plane, we denote by $p(u)=(x_u,y_u)$ the position of vertex $u \in V$ of $G$ in $\Gamma(G)$. For a pair of vertices $u,v\in V$ of $G$, the unit-length vector from $p(u)$ to $p(v)$ is denoted as $\overrightarrow{p(u)p(v)}$, while the line segment delimited by $p(u)$ and $p(v)$ as $\overline{p(u)p(v)}$.


\section{Description of our Heuristic Approach}
\label{sec:algorithm}

In this section, we describe our heuristic algorithm for obtaining drawings with high crossing resolution. Without loss of generality, we assume that the input of our algorithm consists of a graph $G$ and an initial drawing $\Gamma_0$ of $G$ that is of some crossing resolution $c(\Gamma_0)$. We may assume that no two edges of $G$ overlap in $\Gamma_0$, that is, $c(\Gamma_0)>0$. Note that a drawing $\Gamma_0$ of $G$ meeting these preconditions can be, e.g., a circular drawing of $G$ or a drawing obtained by applying a standard spring embedding algorithm on $G$. 

Our algorithm is iterative and at each iteration step performs some operations that are mainly based on randomization. More precisely, at the $i$-th iteration step, we assume that we have computed a drawing $\Gamma_{i-1}$ of some crossing resolution $c(\Gamma_{i-1}) \geq c(\Gamma_0)$, where $\Gamma_0$ is our initial drawing. In other words, we assume, as an invariant property of our algorithm, that the crossing resolution cannot be decreased at some iteration step. Then, a vertex of $\Gamma_{i-1}$ is chosen arbitrary at random from a so-called \emph{vertex-pool}, which may contain:

\begin{itemize}
\item either all the vertices of drawing $\Gamma_{i-1}$, or
\item a prespecified subset of the vertices of drawing $\Gamma_{i-1}$, which we call \emph{critical}.
\end{itemize}

To formally define the critical vertices, we first need to introduced the notion of critical edge-pairs. A pair of edges, say $e$ and $e'$, is called \emph{critical} in drawing $\Gamma_{i-1}$, if $e$ and $e'$ cross in $\Gamma_{i-1}$ and the minimum angle that is formed at their crossing point is either equal (or very close, in some cases) to the crossing resolution $c(\Gamma_{i-1})$ of drawing $\Gamma_{i-1}$. The set of critical vertices of drawing $\Gamma_{i-1}$ is then defined by the four endvertices of each critical edge-pair.  

Intuitively, the critical vertices are the endpoints of the edges that define the crossing resolution of drawing $\Gamma_{i-1}$. As a result, their role in our algorithm is crucial. The reason is that by slightly changing the location of a critical vertex appropriately, one naturally expects to improve the crossing resolution of the current drawing. In fact, what quickly we realized from our experimental evaluation, is that by appropriately changing the location of a critical vertex at each iteration of our algorithm, the crossing resolution of the initial drawing improves rapidly during the first iterations of the algorithm. However, by focusing only at the critical vertices of the drawing, it is highly possible that the algorithm will get trapped to some local minima after a number of iterations. Hence, special care is needed in order to avoid these bottlenecks, especially when the graph to be drawn is large. More precisely, in such a case an algorithm that is based on randomly selecting a vertex to move, so to improve the crossing resolution, will need a large number of iterations in order to converge to a good solution, because it is simply very unlike to select one of the critical vertices to move. We discuss heuristics to avoid such bottlenecks in  Section~\ref{ssec:minima}.

So far, we have described the main idea of our iterative algorithm, which at each iteration step chooses uniformly at random a vertex from the vertex-pool of the current drawing to move, so to improve the crossing resolution. We have not described, however, how the chosen vertex is moved, i.e., how we compute its new position in the next drawing.  

Let $v_i$ be the vertex that has been chosen from the vertex-pool of the drawing $\Gamma_{i-1}$ at the $i$-th iteration of our algorithm. Recall that the crossing resolution $c(\Gamma_{i})$ of drawing $\Gamma_{i}$ that is obtained after the $i$-th iteration of the algorithm must be at least as large as the crossing resolution $c(\Gamma_{i-1})$ of drawing $\Gamma_{i-1}$ (by the invariant of our algorithm), that is, $c(\Gamma_i) \ge c(\Gamma_{i-1})$. To compute the position of vertex $v_i$ in drawing $\Gamma_i$, we consider a set of $\rho$ rays $r_0,r_1,\ldots,r_{\rho-1}$ that all emanate from the point $p(v_i)$ of vertex $v_i$ in drawing $\Gamma_{i-1}$, such that the angle formed by ray $r_j$, with $j=0,1,\ldots,\rho-1$, and the horizontal axis equals to $2j\pi/\rho$, where $\rho>0$ is an integer parameter of the algorithm. These rays are then rotated by an angle that is chosen uniformly at random in the interval $[0,2\pi]$. The position of vertex $v_i$ in drawing $\Gamma_i$ will eventually be along one of the rays $r_0,r_1,\ldots,r_{\rho-1}$. More precisely, for each ray $r_i$ we chose a distance value $\delta_i$ uniformly at random from the interval $[0,\delta]$, where $\delta$ is also a parameter of the algorithm. For each $j=0,1,\ldots,\rho-1$, a new point $\pi_j$ on the plane is obtained translating $p(u)$ along ray $r_j$ at a distance $\delta_j$; we say that point $\pi_j$ is \emph{feasible}, if the crossing resolution of the drawing obtained by placing vertex $v_i$ at point $\pi_j$ and by keeping all other vertices of $G$ in the same positions as in drawing $\Gamma_{i-1}$ is at least as large as the crossing resolution of drawing $\Gamma_{i-1}$\todo{Add a figure illustrating the approach.}. 

If none of the points $\pi_j$, with $j=0,1,\ldots,\rho-1$ is feasible, then the new position of vertex $v_i$ in drawing $\Gamma_i$ is $p(v_i)$, that is, the same as in drawing $\Gamma_{i-1}$, since the crossing resolution of drawing $\Gamma_i$ must be at least as large as the one of $\Gamma_{i-1}$. On the other hand, if there is one or more feasible points (among $\pi_0,\pi_1,\ldots,\pi_{\rho-1}$), then one may consider two different approaches to determine the new position of vertex $v_i$ in drawing $\Gamma_i$. The most natural is to chose this feasible point that maximizes the crossing resolution of the obtained drawing. As an alternative, one may rely again on randomization and chose uniformly at random one the feasible points as the new position of vertex $v_i$ in drawing $\Gamma_i$. In our implementation, we did not observe any significant difference in these two approaches (in terms of the crossing resolution of the obtained drawings), so we simply adopted the first one. In Section~\ref{ssec:termination}, we describe conditions  The iteration steps
 

\begin{itemize}
\item \st{Description of the randomized algorithm}.
\item \st{Big graphs $\rightarrow$ focus on critical edges}.
\item Avoiding local minima.
\item Crossing calculation. 
\end{itemize}

\subsection{Avoiding local minima.}
\label{ssec:minima}

From the above description, it is not difficult to see that our algorithm in a sense mimics the way a human would try to increase the crossing resolution of a drawing; first, one would try to identify...

\subsection{Conditions for termination.}
\label{ssec:termination}

\subsection{Grid drawings: an interesting variant.}
\label{ssec:grid}


\section{Experimental Evaluation}
\label{sec:experiments}

\begin{itemize}
\item Experimental setup.
\item Benchmarks + degree-3.
\item Aesthetics:

\begin{itemize}
\item Number of crossings (\# of crossings).
\item Average size of crossing angles (angle size).
\item Average of edge lengths (edge length).
\item Angular resolution (angular res.).
\end{itemize}

\item Results.
\item Discussion.
\end{itemize}

\section{Conclusions}
\label{sec:conclusions}

\bibliographystyle{abbrvurl}
\bibliography{references}

\end{document}
