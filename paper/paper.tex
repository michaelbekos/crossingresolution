\documentclass{llncs}
\usepackage{amsmath,amssymb}
\usepackage{graphicx,rotating}
\usepackage[font=small]{subfig,caption}
\usepackage{algorithm2e}
\usepackage{todonotes,lineno,paralist}
\usepackage{placeins,rotating,hyperref}
\graphicspath{{figures/}}

\author{Michael~A.~Bekos, Henry~F\"orster, Christian-Marius~Geckeler, Michael~Kaufmann, Amad\"aus~Spallek, Jan~Splett}
\title{A Heuristic Approach towards Drawings of Graph with High Crossing Resolution}

\institute{
Wilhelm-Schickhard-Institut f\"ur Informatik, Universit\"at T\"ubingen, Germany\\
\texttt{\{bekos,foersth,mk\}@informatik.uni-tuebingen.de}\\
\texttt{\{christian-marius.geckeler,amadaeus.spallek,jan.splett\}@student.uni-tuebingen.de}
}

% ============================================================
\begin{document}
\maketitle
% ==================================================================

\begin{abstract}

\end{abstract}

\section{Introduction}
\label{sec:introduction}

In Graph Drawing, there exist a really reach literature and a wide range of techniques for drawing planar graphs; see, e.g.,~\cite{DBLP:books/ph/BattistaETT99,DBLP:conf/dagstuhl/1999dg}. However, the task of drawing a non-planar graph, and in particular when it does not have some special structure (e.g., degree restricted), is difficult and challenging, and as a result the proposed techniques are significantly fewer. 

\section{Related Work}
\label{sec:relatedwork}

\section{Description of our Heuristic Approach}
\label{sec:algorithm}


\section{Experimental Evaluation}
\label{sec:experiments}


\section{Conclusions}
\label{sec:conclusions}

\bibliographystyle{abbrvurl}
\bibliography{references}

\end{document}
