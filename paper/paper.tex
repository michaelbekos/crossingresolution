\documentclass{llncs}
\usepackage{amsmath,amssymb}
\usepackage{graphicx,rotating}
\usepackage[font=small]{subfig,caption}
\usepackage{algorithm2e}
\usepackage{todonotes,lineno,paralist}
\usepackage{placeins,rotating,hyperref}
\graphicspath{{figures/}}

\author{Michael~A.~Bekos, Henry~F\"orster, Christian-Marius~Geckeler, Michael~Kaufmann, Amad\"aus~Spallek, Jan~Splett}
\title{A Heuristic Approach towards Drawings of Graphs with High Crossing Resolution}

\institute{
Wilhelm-Schickhard-Institut f\"ur Informatik, Universit\"at T\"ubingen, Germany\\
\texttt{\{bekos,foersth,mk\}@informatik.uni-tuebingen.de}\\
\texttt{\{christian-marius.geckeler,amadaeus.spallek,jan.splett\}@student.uni-tuebingen.de}
}

% ============================================================
\begin{document}
\maketitle
% ==================================================================

\begin{abstract}
The \emph{crossing resolution} of a non-planar drawing of a graph is the minimum angle formed at the crossing points of the edges. Recent experiments have shown that the larger the crossing resolution is, the easier it is to read and interpret a drawing of a graph. However, maximizing the crossing resolution turns out to be an NP-hard problem in general and only heuristic algorithms are known that are mainly based on appropriately adjusting spring embedding algorithms. 
 
In this paper, we propose a new --randomization-based-- heuristic algorithm for the crossing resolution maximization problem and we experimentally compare it again the known ones from the literature. Our experimental evaluation indicates that the new heuristic significantly outperforms the known ones in terms of the quality of the produced layouts (in terms of the actual crossing resolution), but this comes at the cost of slightly higher running time, especially for large and dense graphs. 
\end{abstract}

\section{Introduction}
\label{sec:introduction}

In Graph Drawing, there exist a really reach literature and a wide range of techniques for drawing planar graphs; see, e.g.,~\cite{DBLP:journals/combinatorica/FraysseixPP90,DBLP:conf/gd/GutwengerM98,DBLP:journals/algorithmica/Kant96}. However, drawing a non-planar graph, and in particular when it does not have some special structure (e.g., degree restricted), is a difficult and challenging task, mainly due to the edge crossings which negatively affect the drawing's quality~\cite{DBLP:journals/iwc/Purchase00}. As a result, the proposed techniques are significantly fewer (e.g., crossing minimization heuristics~\cite{DBLP:journals/algorithmica/EadesW94,DBLP:journals/tsmc/SugiyamaTT81}, energy-based layout algorithms~\cite{DBLP:journals/congnum/Eades84,DBLP:journals/spe/FruchtermanR91}); for an overview refer to~\cite{DBLP:books/ph/BattistaETT99,DBLP:conf/dagstuhl/1999dg,DBLP:reference/crc/2013gd}.

In this context, Huang et al.~\cite{DBLP:conf/apvis/Huang07,DBLP:journals/vlc/HuangEH14} a decade ago introduced some important statistical evidence (through a series of eye-tracking experiments), that edge crossings may not negatively affect the drawing's quality too much (and therefore the human's ability to read and interpret it), when the angles formed by the crossing edges are large. In other words, while prior to these experiments it was commonly accepted that mainly the number of crossings is the most important parameter for judging the quality and readability of a non-planar graph drawing, it turned out that the types of edge crossings also matter. As a result, a new and prominent research direction was initiated, which is unofficially recognized under the term ``beyond planarity''~\cite{Shonan2016,Dagstuhl2016,SoCG2017} and mainly focuses on graphs and their properties, when different constraints on the types of edges crossings are imposed; refer to~\cite{DBLP:journals/jgaa/BekosKM18,DBLP:journals/corr/abs-1804-07257} for recent surveys. 

Formally, the minimum angle formed by any two crossed edges in a drawing is refer to as its \emph{crossing resolution}. Analogously, the crossing resolution of a graph is defined as the maximum crossing resolution over all its drawings. Clearly, the crossing resolution of a non-planar graph is at most $90^\circ$, while a graph that admits a drawing with crossing resolution $90^\circ$ is refer to as \emph{right-angle-crossing} graph or \emph{RAC} graph, for short. For these graphs, several results, mostly of theoretical nature, are known (refer to Section~\ref{sec:relatedwork} for a short overview). Notably, RAC graph are sparse (they contain at most $4n-10$ edges~\cite{DBLP:journals/tcs/DidimoEL11}, where $n$ denotes the number of  vertices), while deciding whether a graph is RAC is NP-hard~\cite{DBLP:journals/jgaa/ArgyriouBS12}. 

The latter result is already an indication that the crossing resolution maximization problem might also be difficult, even though, formally, its complexity has not been settled yet for values of the crossing resolution smaller than $90^\circ$. Also, the literature is significantly more limited, when restricting the crossing resolution to be smaller than $90^\circ$, as also evidenced by Section~\ref{sec:relatedwork}. 

From a practical point of view, we are only aware of two methods that aim in drawings with high crossing resolution; both of them are adjustments of spring embedding algorithms. The first one is due to Huang, Eades, Hong and Lin~\cite{DBLP:journals/vlc/HuangEHL13}, while the second one due to Argyriou, Bekos and Symvonis~\cite{DBLP:journals/cj/ArgyriouBS13}. Common in both algorithms are appropriate forces, which are applied on the endvertices of every pair of crossing edges. Of course, each of the two algorithms uses a different way to compute (the direction and the magnitude of) the forces, but the underlying idea in both algorithms is the same: the smaller the corresponding crossing angles are, the larger are the magnitudes of the forces applied at their endvertices. 

In this paper, we attack the crossing resolution maximization problem by following a completely different approach. We suggest a simple and quite intuitive randomization method for computing drawings with high crossing resolution, which in practise turned out to be quite efficient\footnote{Note that, in general, randomization is a technique that has not been deeply examined in Graph Drawing, as it seems to be difficult to computed the expected quality of the produced drawings.}. However, since we could not provide any theoretical guarantee on the expected quality of the produced drawings (mainly due to the nature of the problem itself), we decided to follow a more practical approach. To this end, we implemented our algorithm and the ones presented in~\cite{DBLP:journals/vlc/HuangEHL13} and~\cite{DBLP:journals/cj/ArgyriouBS13}, and we experimentally compared the crossing resolutions of the drawings produced by them for standard benchmark graphs (such as the Rome and the North graphs) that are widely used in Graph Drawing for comparing different drawing algorithms. Our experimental evaluation indicates that our new method significantly outperforms the ones that are based on adjustments of spring embedding algorithms~\cite{DBLP:journals/vlc/HuangEHL13,DBLP:journals/cj/ArgyriouBS13} in terms of crossing resolution, but this comes at the cost of slightly higher running time, especially for large and dense graphs. 

The remainder of this paper is structured as follows. Section~\ref{sec:relatedwork} overviews related results. In Section~\ref{sec:preliminaries}, we introduce preliminary notions and notation used throughout the paper. Our algorithm is presented in detail in Section~\ref{sec:algorithm} and is experimentally evaluated against the ones of Huang et al.~\cite{DBLP:journals/vlc/HuangEHL13} and Argyriou at al.~\cite{DBLP:journals/cj/ArgyriouBS13} in Section~\ref{sec:experiments}. We conclude in Section~\ref{sec:conclusions} with open problems and future direction.
 
\section{Related Work}
\label{sec:relatedwork}

\section{Preliminaries}
\label{sec:preliminaries}

\section{Description of our Heuristic Approach}
\label{sec:algorithm}


\section{Experimental Evaluation}
\label{sec:experiments}


\section{Conclusions}
\label{sec:conclusions}

\bibliographystyle{abbrvurl}
\bibliography{references}

\end{document}
