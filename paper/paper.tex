\documentclass[runningheads]{llncs}
\usepackage{amsmath,amssymb}
\usepackage{graphicx,rotating}
\usepackage[font=small]{subfig,caption}
\usepackage{algorithm2e}
\usepackage{todonotes,lineno,paralist,soul}
\usepackage{placeins,rotating,hyperref}
\graphicspath{{figures/}}


\newcommand{\myparagraph}[1]{\medskip\noindent\textbf{#1}.}
\author{Michael~A.~Bekos, Henry~F\"orster, Christian~Geckeler, Lukas Holl\"ander, Michael~Kaufmann, Amad\"aus~M.~Spallek, Jan~Splett}

\authorrunning{M.~A.~Bekos et al.}
\title{A Heuristic Approach towards Drawings of Graphs with High Crossing Resolution}
\titlerunning{Drawings of Graphs with High Crossing Resolution}

\institute{
Wilhelm-Schickhard-Institut f\"ur Informatik, Universit\"at T\"ubingen, Germany\\
\texttt{\{bekos,foersth,mk\}@informatik.uni-tuebingen.de}\\
\texttt{\{christian-marius.geckeler,jan-lukas.hollaender,amadaeus.spallek,jan.splett\} @student.uni-tuebingen.de}
}

% ==================================================================
\begin{document}
\maketitle
\linenumbers
% ==================================================================

\begin{abstract}
The \emph{crossing resolution} of a non-planar drawing of a graph is the value of the minimum angle formed by any pair of crossing edges. Recent experiments have shown that the larger the crossing resolution is, the easier it is to read and interpret a drawing of a graph. However, maximizing the crossing resolution turns out to be an NP-hard problem in general and only heuristic algorithms are known that are mainly based on appropriately adjusting force-directed algorithms. 
 
In this paper, we propose a new heuristic algorithm for the crossing resolution maximization problem and we experimentally compare it against the known approaches from the literature. Our experimental evaluation indicates that the new heuristic produces drawings with better crossing resolution, but this comes at the cost of slightly higher aspect ratio, especially when the input graph is large. 
\end{abstract} 

% ==================================================================
\section{Introduction}
\label{sec:introduction}
% ==================================================================

In Graph Drawing, there exists a really rich literature and a wide range of~techniques for drawing planar graphs; see, e.g.,~\cite{DBLP:journals/combinatorica/FraysseixPP90,DBLP:conf/gd/GutwengerM98,DBLP:journals/algorithmica/Kant96}. However, drawing a non-planar graph, and in particular when it does not have some special structure (e.g., degree restriction), is a difficult and challenging task, mainly due to the~edge crossings that negatively affect the drawing's quality~\cite{DBLP:journals/iwc/Purchase00}. As a result, the established techniques are significantly fewer (e.g., crossing minimization heuristics \cite{DBLP:journals/algorithmica/EadesW94,DBLP:journals/tsmc/SugiyamaTT81}, energy-based layout algorithms~\cite{DBLP:journals/congnum/Eades84,DBLP:journals/spe/FruchtermanR91}); for an overview refer to~\cite{DBLP:books/ph/BattistaETT99,DBLP:conf/dagstuhl/1999dg,DBLP:reference/crc/2013gd}.

In this context, Huang et al.~\cite{DBLP:conf/apvis/Huang07,DBLP:journals/vlc/HuangEH14} a decade ago introduced some important experimental evidence,
% (through eye-tracking experiments), 
that edge crossings may not negatively affect the drawing's quality too much (and hence the human's ability to read and interpret it), when the angles formed by the crossing edges are large. In other words, while prior to these experiments it was commonly accepted that mainly the number of crossings is the most important parameter for judging the quality of a non-planar graph drawing, it turned out that the types of edge crossings also matter. As a result, a new and prominent research direction was initiated, recognized under the term ``beyond planarity''~\cite{Shonan2016,Dagstuhl2016,SoCG2017}, which focuses on graphs and their properties, when different constraints on the types of edges crossings are imposed; refer to~\cite{DBLP:journals/corr/abs-1804-07257} for a recent survey. 

Formally, the value of the minimum angle formed by any two crossing edges in a drawing is refered to as its \emph{crossing resolution}; the crossing resolution of a graph is defined as the maximum crossing resolution over all its drawings. Clearly, the crossing resolution of a non-planar graph is at most $90^\circ$, while a graph that admits a drawing with crossing resolution $90^\circ$ is called \emph{right-angle-crossing} graph or \emph{RAC} graph, for short; see Figure~\ref{fig:examples}. For these graphs, several results, mostly of theoretical nature, are known (refer to Section~\ref{sec:relatedwork} for a short overview). Notably, RAC graph are sparse (they contain at most $4n-10$ edges~\cite{DBLP:journals/tcs/DidimoEL11}, where $n$ denotes the number of  vertices), while deciding whether a graph is RAC is NP-hard~\cite{DBLP:journals/jgaa/ArgyriouBS12}.

\begin{figure}[t!]
	\centering
	\subfloat[\label{fig:k5} {}]{
	\includegraphics[page=1,scale=0.85]{figures/examples}}
	\hfil
	\subfloat[\label{fig:k6} {}]{
	\includegraphics[page=2,scale=0.85]{figures/examples}}
	\caption{%
	(a)~A RAC drawing of the complete graph $K_5$, and
	(b)~a drawing of the complete graph $K_6$, whose crossing resolution is arbitrarily close to $90^\circ$.}
	\label{fig:examples}
\end{figure} 

The latter result is already an indication that the problem of finding drawings with high crossing resolution might also be difficult, even though, formally, its complexity has not been settled yet for values of the crossing resolution smaller than $90^\circ$. Also, the literature is significantly more limited, when restricting the crossing resolution to be smaller than $90^\circ$, as also evidenced by Section~\ref{sec:relatedwork}. 

From a practical point of view, we are only aware of two methods that aim at drawings with high crossing resolution; both of them are adjustments of force-directed algorithms~\cite{DBLP:journals/congnum/Eades84}. The first one is due to Huang et al.~\cite{DBLP:journals/vlc/HuangEHL13}, while the~second one is due to Argyriou et al.~\cite{DBLP:journals/cj/ArgyriouBS13}. Common in both algorithms is that they apply appropriate forces on the endvertices of every pair of crossing edges. Each of them uses a different way to compute (the direction and the magnitude of) the forces, but the underlying idea of both is the same: the smaller the crossing angles are, the larger are the magnitudes of the forces applied at their endvertices. 

In this work, we approach the crossing resolution maximization problem from a different perspective. We suggest a simple and intuitive randomization method,
% for computing drawings with high crossing resolution, 
which, in a sense, mimics the way a human would try to increase the crossing resolution of a drawing. How would one increase the crossing resolution of a given drawing? First, she would try to identify the pair of edges that define the crossing resolution of the drawing (we call them \emph{critical} edges); then, she would try to move an endvertex of this pair (which we choose at random), hoping that by this move the crossing resolution will increase. Of course, we cannot consider all possible positions for the vertex to be moved. Instead, we consider a small set of randomly generated ones. If there exists a position among them, that does not lead to a reduction of the crossing resolution, we move the vertex to this~position.

In general, randomization is a technique that has not been deeply examined in Graph Drawing, as it seems difficult to even speculate about the expected quality of the produced drawings; a notable exception is the randomized approach by Goldschmidt and Takvorian~\cite{DBLP:journals/networks/GoldschmidtT94} for computing large planar subgraphs. Since we also could not provide any theoretical guarantee on the expected quality of the produced drawings, 
%(mainly due to the nature of the problem itself), 
we followed a more practical approach. We implemented our algorithm and the force-directed ones of~\cite{DBLP:journals/cj/ArgyriouBS13} and~\cite{DBLP:journals/vlc/HuangEHL13}, and we experimentally compared them on standard benchmark graphs.
% (such as the Rome and the North graphs) that are widely used in Graph Drawing for comparing different drawing algorithms. 
Our evaluation indicates that our method significantly outperforms the force-directed ones~\cite{DBLP:journals/cj/ArgyriouBS13,DBLP:journals/vlc/HuangEHL13} in terms of crossing resolution, but this comes at the cost of slightly worse running time
%, especially 
for large and dense graphs. Analogous results are obtained, when our algorithm and the ones of~\cite{DBLP:journals/cj/ArgyriouBS13} and~\cite{DBLP:journals/vlc/HuangEHL13} are adjusted to maximize  the \emph{angular resolution} (i.e., the minimum value of the angle between any two adjacent edges~\cite{DBLP:journals/siamcomp/FormannHHKLSWW93}) or the \emph{total resolution} (i.e., the minimum of the angular and the crossing resolution~\cite{DBLP:journals/cj/ArgyriouBS13}).

\medskip\noindent{\it Preliminaries:}
Unless otherwise specified, in this paper we consider simple undirected graphs. Let $G=(V,E)$ be such a graph. The degree of vertex $u\in V$ of $G$ is denoted by $d(u)$. The degree $d(G)$ of  graph $G$ is defined as the maximum degree of its vertices, i.e., $d(G)=\max_{u\in V}d(u)$.
%
Given a drawing $\Gamma(G)$ of $G$, we denote by $p(u)=(x_u,y_u)$ the position of vertex $u \in V$ of $G$ in $\Gamma(G)$. %For a pair of vertices $u,v\in V$ of $G$, the unit-length vector from $p(u)$ to $p(v)$ is denoted as $\overrightarrow{p(u)p(v)}$, while the line segment delimited by $p(u)$ and $p(v)$ as $\overline{p(u)p(v)}$.

\medskip\noindent{\it Structure of the paper:}
The remainder of this paper is structured as follows. Section~\ref{sec:relatedwork} overviews related works. 
%In Section~\ref{sec:preliminaries}, we introduce preliminary notions and the notation used throughout the paper. 
Our algorithm is presented in detail in Section~\ref{sec:algorithm} and is experimentally evaluated against the ones of Huang et al.~\cite{DBLP:journals/vlc/HuangEHL13} and Argyriou et al.~\cite{DBLP:journals/cj/ArgyriouBS13} in Section~\ref{sec:experiments}. We conclude in Section~\ref{sec:conclusions} with open problems.
 
% ==================================================================
\section{Related Work}
\label{sec:relatedwork}
% ==================================================================

As already mentioned, the study of the crossing resolution maximization problem has mainly focused on its optimal case, i.e., on the study of RAC graphs. An $n$-vertex RAC graph has at most $4n-10$ edges~\cite{DBLP:journals/tcs/DidimoEL11}, while deciding whether a graph is RAC is NP-hard~\cite{DBLP:journals/jgaa/ArgyriouBS12}. The maximally-dense RAC graphs are 1-planar~\cite{DBLP:journals/dam/EadesL13}, i.e., they can be drawn with at most one crossing per edge. Actually, several~relationships between the class of RAC graphs and subclasses of 1-planar graphs are known~\cite{DBLP:journals/dam/BachmaierBHNR17,DBLP:journals/tcs/BrandenburgDEKL16}. Deciding, however, whether a 1-planar graph is RAC is NP-hard~\cite{DBLP:journals/tcs/BekosDLMM17}. Note that the problem of finding RAC drawings has also been studied in the presence of bends~\cite{DBLP:journals/jgaa/AngeliniCDFBKS11,DBLP:journals/comgeo/ArikushiFKMT12,DBLP:journals/tcs/DidimoEL11,DBLP:journals/mst/GiacomoDLM11} and by imposing restrictions on the degree~\cite{DBLP:conf/s-egc/AngeliniBDFHKLL11}, the structure~\cite{DBLP:journals/ipl/DidimoEL10} and the drawing~\cite{DBLP:journals/algorithmica/GiacomoDEL14,DBLP:conf/wg/HongN15} of the graph. The results are fewer, when the right-angle constraint is relaxed. Dujmovic et al.~\cite{DBLP:journals/cjtcs/DujmovicGMW11} proved that an $n$-vertex graph with crossing resolution at least $\alpha$ radians, has at most $(3n-6)\pi/\alpha$ edges. Corresponding density results are also known in the presence of bends~\cite{DBLP:journals/siamdm/AckermanFT12,DBLP:journals/mst/GiacomoDLM11}.

An immediate observation emerging from the above overview is that the~focus has been primarily on theoretical aspects of the problem. Most of the approaches that could be useful in practise are based on force-directed techniques~\cite{DBLP:books/ph/BattistaETT99,DBLP:journals/congnum/Eades84}. COWA is a system that supports conceptual web site traffic analysis~\cite{DBLP:conf/apvis/DidimoLR10}; its algorithmic core is a force-directed heuristic to compute simultaneous embeddings of two non-planar graphs with high crossing resolution. 
%
Didimo et al.~\cite{DBLP:conf/gd/DidimoLR10} describe topology-driven force-directed heuristics to achieve good trade-offs in terms of number of edge crossings, crossing resolution, and geodesic edge tendency; the obtained drawings, however, are not straight-line. 
%
For straight-line drawings, Nguyen et al.~\cite{DBLP:conf/gd/NguyenEHH10} suggest a quadratic-program to increase the crossing angles of circular drawings. 
%
Of more general scope are the already mentioned force-directed algorithms of Argyriou et al.~\cite{DBLP:journals/cj/ArgyriouBS13} and Huang et al.~\cite{DBLP:journals/vlc/HuangEHL13}.


%As already mentioned, the study of the crossing resolution maximization problem has mainly focused on (properties of) RAC graphs, that is, on the optimal case of the crossing resolution maximization problem. Their study was initiated by Didimo et al.~\cite{DBLP:journals/tcs/DidimoEL11}, who showed that an $n$-vertex RAC graph has at most $4n-10$ edges. Angelini et al.~\cite{DBLP:journals/jgaa/AngeliniCDFBKS11} presented results for two interesting variants, in which the input graph is either an acyclic digraph or a graph of bounded vertex degree. Argyriou et al.~\cite{DBLP:journals/jgaa/ArgyriouBS12} showed that the problem of deciding whether a graph is RAC is NP-hard. Angelini et al.~\cite{DBLP:conf/s-egc/AngeliniBDFHKLL11} observed that planar graphs of bounded degree admit straight-line RAC drawings in subquadratic area. Didimo et al.~\cite{DBLP:journals/ipl/DidimoEL10} characterized the complete bipartite RAC graphs. Di~Giacomo et al.~\cite{DBLP:journals/algorithmica/GiacomoDEL14}, and Hong and Nagamochi~\cite{DBLP:conf/wg/HongN15} studied variants, in which the vertices of the input graph are restricted on two parallel lines and on the circumference of a circle, respectively. Eades and Liotta~\cite{DBLP:journals/dam/EadesL13} showed that the optimal RAC graphs (i.e., those of maximum density) are 1-planar (i.e., they admit drawings in which each edge is crossed at most once). Deciding, however, whether a 1-planar graph is RAC is NP-hard~\cite{DBLP:journals/tcs/BekosDLMM17}. Other relationships between the class of RAC graphs and subclasses of 1-planar graphs have also been studied by Bachmaier et al.~\cite{DBLP:journals/dam/BachmaierBHNR17} and Brandenburg et al.~\cite{DBLP:journals/tcs/BrandenburgDEKL16}. Note that the problem of finding RAC drawings has been also studied in the presence of bends; see e.g.~\cite{DBLP:journals/jgaa/AngeliniCDFBKS11,DBLP:journals/comgeo/ArikushiFKMT12,DBLP:journals/tcs/DidimoEL11,DBLP:journals/mst/GiacomoDLM11}.  

%To the best of our knowledge, there is only one work, by Dujmovic et al.~\cite{DBLP:journals/cjtcs/DujmovicGMW11}, which studies the crossing resolution maximization problem by relaxing the right-angle constraint on the crossing angles of the computed drawings. More precisely, in their work, Dujmovic et al.~\cite{DBLP:journals/cjtcs/DujmovicGMW11} proved that an $n$-vertex graph, whose crossing resolution is at least $\alpha$ (in radians) has at most $(3n-6)\pi/\alpha$ edges. Corresponding density results for the case, in which few bends are allowed along each edge, are known by Ackerman et al.~\cite{DBLP:journals/siamdm/AckermanFT12} and Di Giacomo et al.~\cite{DBLP:journals/mst/GiacomoDLM11}.  



%An immediate observation emerging from the above literature overview is that the focus of the study of the crossing resolution maximization problem has been primarily on theoretical aspects of the problem. Most of the approaches that could be useful in practise are based on adjustments of force-directed techniques~\cite{DBLP:journals/congnum/Eades84}, according to which a graph is modelled as a physical system with forces acting on it, and a (good) drawing is obtained by an equilibrium of the system; for an introduction and a discussion of several variants refer to~\cite{DBLP:books/ph/BattistaETT99}. %Actually, the absence of approaches for computing drawings with high crossing resolution in practice is rather obvious and it can be justified, up to a certain point, by the fact that the problem turns to be NP-hard, when asking for drawings with optimal crossing resolution. 
%
%More concretely, from an applicative point of view, Didimo et al.~\cite{DBLP:conf/apvis/DidimoLR10} describe a system, called \emph{COWA}, to support conceptual web site traffic analysis, whose algorithmic core is a force-directed heuristic to compute simultaneous embeddings of two non-planar graphs with high crossing resolution. 
%
%In a follow up work, Didimo et al.~\cite{DBLP:conf/gd/DidimoLR10} describe heuristics, designed within the topology-driven force-directed framework, to achieve good trade-offs in terms of number of edge crossings, crossing resolution, and geodesic edge tendency. 
%
%However, the obtained drawings are not straight-line. For straight-line drawings, Nguyen et al.~\cite{DBLP:conf/gd/NguyenEHH10} suggest a quadratic-programming based approach to increase the crossing angles of circular drawings. 
%
%Of more general scope are the already mentioned works of Huang et al.~\cite{DBLP:journals/vlc/HuangEHL13} and Argyriou et al.~\cite{DBLP:journals/cj/ArgyriouBS13}, which are also based on the force-directed technique. 



% ==================================================================
\section{Description of our Heuristic Approach}
\label{sec:algorithm}
% ==================================================================

In this section, we describe our heuristic for obtaining drawings with high crossing resolution. The input of our heuristic consists of a graph $G$ and an initial drawing $\Gamma_0$ of $G$ with crossing resolution $c(\Gamma_0)$. We assume that no two edges of~$G$ overlap in $\Gamma_0$, i.e., $c(\Gamma_0)>0$. A circular drawing or a drawing obtained by applying a force-directed algorithm on $G$ clearly meets this precondition. 

Our algorithm is iterative and at each iteration performs some operations that are mainly based on randomization. At the $i$-th iteration, we assume that we have computed a drawing $\Gamma_{i-1}$ of crossing resolution $c(\Gamma_{i-1}) \geq c(\Gamma_0)$. 
%, where $\Gamma_0$ is our initial drawing. 
In other words, we assume, as an invariant for our algorithm, that the crossing resolution cannot be decreased at some iteration. Then, a vertex of $\Gamma_{i-1}$ is chosen arbitrarily at random based on the so-called \emph{vertex-pool}, which may contain:
%
%\begin{itemize}
\begin{inparaenum}[(i)]
\item either all vertices of $\Gamma_{i-1}$, or
\item a prespecified subset of the vertices of $\Gamma_{i-1}$, called \emph{critical}.
\end{inparaenum}

Intuitively, the critical vertices are the endpoints of the edges that define the crossing resolution of drawing $\Gamma_{i-1}$. To formally define them, we first need to introduce the notion of critical edge-pairs. A pair of edges $e$ and $e'$ is called \emph{critical} in $\Gamma_{i-1}$, if $e$ and $e'$ cross in $\Gamma_{i-1}$ and the minimum angle that is formed at their crossing point is equal to $c(\Gamma_{i-1})$. The set of critical vertices of $\Gamma_{i-1}$ is then defined by the four endvertices of each critical edge-pair.  

The role of critical vertices is central in our algorithm
%\footnote{If the focus is not on the critical vertices for a large graph, then an algorithm that is based on randomly selecting a vertex to move, so to improve the crossing resolution, will need a large number of iterations to converge to a good solution, because it is very unlike to select one of them to move.}
\footnote{If the focus is not on the critical vertices for a large graph, then our algorithm will need a large number of iterations to converge to a good solution, because it is simply very unlikely to select to move one of the vertices that define the crossing resolution.}: By appropriately changing the location of a critical vertex or of a vertex in the neighbourhood of the critical vertices, we naturally expect to improve the crossing resolution of the current drawing. We turned this observation into an algorithmic implementation through a weighted random selection procedure, so that the vertices at distance~$i$ from the ones of the vertex-pool have higher weights than the corresponding ones at distance $j$  in the graph, when $0 \leq i<j$. So, if the vertex-pool contains critical vertices, then the closer a vertex is to the critical vertices, the more likely it is to be chosen. Otherwise, each vertex can be chosen with the same probability.

%With this procedure, what we quickly realized from our experimental evaluation, is that by appropriately changing the location of a critical vertex at each iteration of our algorithm, the crossing resolution of the initial drawing improves rapidly during the first iterations of the algorithm. 
What we quickly realized from our practical analysis, is that the crossing resolution of the initial drawing improves rapidly during the first iterations of the algorithm. However, by focusing only at the critical vertices, it is~highly possible that the algorithm will get trapped to some local maxima after a number of iterations. So, special care is needed to avoid these bottlenecks, especially when the input graph is large. We discuss ways to avoid them later in this section.

So far, we have described the main idea of our algorithm, which at each iteration chooses uniformly at random a vertex of the current drawing to move (based on the content of the vertex-pool), so to improve the crossing resolution. Next, we  described how to compute its new position in the next drawing.  

Let $v_i$ be the vertex of $\Gamma_{i-1}$ that has been chosen to be moved at the $i$-th iteration. 
%Recall that the crossing resolution $c(\Gamma_{i})$ of drawing $\Gamma_{i}$ that is obtained after the $i$-th iteration of the algorithm must be at least as large as the crossing resolution $c(\Gamma_{i-1})$ of drawing $\Gamma_{i-1}$ (by the invariant of our algorithm), that is, $c(\Gamma_i) \ge c(\Gamma_{i-1})$. 
To compute the position of $v_i$ in the next drawing $\Gamma_i$, we consider a set of $\rho$ rays $r_0,r_1,\ldots,r_{\rho-1}$ that all emanate from $p(v_i)$ in $\Gamma_{i-1}$, such that the angle formed by ray $r_j$, with $j=0,1,\ldots,\rho-1$, and the horizontal axis equals to $2j\pi/\rho$, where $\rho>0$ is an integer parameter of the algorithm. These rays are then rotated by an angle that is chosen uniformly at random in the interval $[0,2\pi]$; see Fig.~\ref{fig:algo}. The position of vertex $v_i$ in $\Gamma_i$ will eventually be along one of the rays $r_0,r_1,\ldots,r_{\rho-1}$. More precisely, for each ray $r_i$ we chose a distance value $\delta_i$ uniformly at random from the interval $[\delta_{min},\delta_{max}]$, where $\delta_{min}$ and $\delta_{max}$ are two positive parameters of the algorithm. For each $j=0,1,\ldots,\rho-1$, a new point $\pi_j$ is obtained by translating $p(u)$ along $r_j$ by a distance $\delta_j$; point $\pi_j$ is \emph{feasible}, if the crossing resolution of the drawing obtained by placing vertex $v_i$ at $\pi_j$ and by keeping all other vertices of $G$ in their positions in $\Gamma_{i-1}$ is at least as large as the crossing resolution of $\Gamma_{i-1}$, and there is no vertex of $\Gamma_{i-1}$ at $\pi_j$. 

\begin{figure}[t!]
	\centering
	\subfloat[\label{fig:algo-rays} {}]{	
	\includegraphics[page=1, scale=0.9]{figures/algorithm}}
	\hfil
	\subfloat[\label{fig:algo-move} {}]{
	\includegraphics[page=2, scale=0.9]{figures/algorithm}}
	\caption{%
	Illustration of an iteration step of our algorithm: 
	(a)~The chosen vertex is the white one; 
	the computed rays $r_0,\ldots,r_7$ have been rotated by $8^\circ$; 
	the black-colored points along these rays are points $\pi_0,\ldots,\pi_7$; 
	among them, $\pi_4$ yields the best solution.
	(b)~The resulting drawing after moving the vertex at position $\pi_2$.}
	\label{fig:algo}
\end{figure} 

If none of the points $\pi_j$, with $j=0,1,\ldots,\rho-1$ is feasible, then the position of $v_i$ in $\Gamma_i$ is $p(v_i)$, i.e., same as in $\Gamma_{i-1}$, since $c(\Gamma_i) \geq c(\Gamma_{i-1})$ must hold. If there is one or more feasible points, then one may consider two different approaches to determine the position of $v_i$ in $\Gamma_i$. The most natural is to chose the feasible point that maximizes the crossing resolution of the obtained drawing. As an alternative, one may rely again on randomization and chose uniformly at random one of the feasible points as the position of $v_i$ in $\Gamma_i$. We note that we did not observe any significant difference between these two approaches (in terms of the crossing resolution of the obtained drawings), so we simply adopted the first one. 

The termination condition of our algorithm is simple and depends on an input parameter~$\tau$. More specifically, if the crossing resolution has not been improved during the last $\tau$ iterations, then we assume that the algorithm has converged and we do not proceed any more.

% ==================================================================
\myparagraph{Avoiding local maxima}
% ==================================================================
%
To avoid getting trapped to locally optimal solutions, we mainly investigated two approaches, which are both  parametrizable by two input parameters $\zeta$ and $\zeta'$. The first mimics the human behaviour. What would~one do to escape from a locally optimal solution? She would stop trying to move the endvertices of the edges defining the crossing resolution; she would rather start moving ``irrelevant'' vertices hoping that 
%by changing the embedding of the graph or its crossing structure, 
by doing so a better solution will be easier to be computed afterwards. Our algorithm is mimicking this idea as follows: 
%
\begin{inparaenum}[(i)]
%\item As long as the algorithm detects that within the last $\zeta$ iterations the crossing resolution of the graph has being improved, the vertex-pool contains only critical vertices. 
\item if during the last $\zeta$ iterations the crossing resolution has not been~improved, then the vertex-pool becomes \emph{wider} containing all the vertices, and the algorithm is executed with this vertex-pool for $\zeta'$ iterations;
\item afterwards, the vertex-pool switches back to the critical vertices.
\end{inparaenum}
%
%From our extensive practical analysis, we quickly realized that, while 
While this approach turned out to be quite effective for medium-size graphs, for larger graphs, unfortunately, it was not so efficient; in most iterations with the wider vertex-pool, the embedding could not change in a beneficial way for the algorithm~to~proceed.

%What we observed is that in larger graphs it was unlikely to change the embedding of the graph in a beneficial way for the algorithm to proceed, mainly because of the random choice of the vertices to be moved; in most iterations with the wider vertex-pool, the vertices chosen to be moved were not in the part of the drawing where the critical vertices resided.

Our second approach is based on parameters $\rho$, $\delta_{min}$ and $\delta_{max}$ of the algorithm. Our idea was that if the algorithm gets trapped to a locally optimal solution, then a ``drastical'' or ``sharp'' move may help to escape. We turned this idea into an algorithmic implementation as follows: 
%
\begin{inparaenum}[(i)]
\item if during the last $\zeta$ iterations the crossing resolution has not been improved, we double the values of $\rho$, $\delta_{min}$ and $\delta_{max}$, and the algorithm is executed with these values for $\zeta'$ iterations;
\item afterwards, $\rho$, $\delta_{min}$ and $\delta_{max}$ switch back to this initial value.
\end{inparaenum}
%
Of course, this approach may lead to drawings with larger area, but this is expected, as it turns out that drawings with high crossing resolution may require large area~\cite{DBLP:journals/jgaa/AngeliniCDFBKS11,DBLP:journals/tcs/BrandenburgDEKL16}. 

% ==================================================================
\myparagraph{Complexity related issues}
% ==================================================================
%
A critical factor that highly affects the efficiency of our algorithm is the computation of the crossing points of the edges (and the corresponding angles at these points, which determine the crossing resolution of the drawing). Given  a drawing, a naive approach to compute its crossings requires $O(m^2)$ time, which can be slightly improved using a standard plane-sweep technique to $O(m \log m + c)$ time, where $m$ and $c$ are the number of edges and crossings of the drawing, respectively; see, e.g.,~\cite{DBLP:books/lib/BergCKO08}. 

However, if the algorithm had to compute all crossing points and the corresponding angles at these points for each candidate position of each iteration, then it would not be useful in practice. Instead, we adopted a different approach, which turned out to be quite efficient in practice. Recall that we denoted by $v_i$ the vertex chosen at the $i$-th iteration step, and by $\pi_0,\ldots,\pi_{\rho-1}$ the candidate points to move $v_i$. Let $e_0,\ldots,e_{d_i-1}$ be the edges incident to $v_i$, where $d_i=deg(v_i)$. Next, for each edge $e_k$ with $k=0,\ldots,d_i-1$  we compute the crossings and the corresponding crossing angles of $e_k$ with all other edges in $\Gamma_{i-1}$. Let $\phi_i$ be the minimum crossing angle computed with this procedure; this is our reference angle. Also, for each candidate position $\pi_j$ with $j=0,\ldots,\rho-1$, and for each edge $e_k$ with $k=0,\ldots,d_i-1$, we compute the crossings and the corresponding crossing angles of $e_k$ with all other edges of the drawing, assuming that $v_i$ is at $\pi_j$. Let $\chi_j$ be the minimum crossing angle computed with this approach, when $v_i$ is at position $\pi_j$. Clearly, $\pi_j$ is feasible only if $\chi_j \geq \phi_i$. Note that the complexity of this approach is $O(deg(v_i) \cdot m)$, which in worst-case is $O(nm)$. 

% ==================================================================
\subsection{Some interesting variants}
\label{ssec:variants}
% ==================================================================

In the remainder of this section, we discuss some interesting variants of our~algorithm, which are motivated by the following preliminary observations that we made from the computed layouts: Drawings with good crossing resolution tend to have bad aspect ratio (that is, the ratio of the longest to the shortest edge is large) and poor angular resolution (that is, the angles formed by adjacent edges are small). The former seems to be a consequence of the fact that drawings with good crossing resolution tend to be quite demanding in area. The later can easily become clear by an example; if in a drawing all edges are either almost horizontally or almost vertically drawn, then the crossing resolution of this drawing is arbitrarily close to $90^\circ$, while its angular resolution is arbitrarily close to~$0^\circ$. 

% ==================================================================
\myparagraph{Aspect ratio}
% ==================================================================
%
Formally, the aspect ratio of a drawing is the ratio of the length of its longest edge to the length of its shortest edge. Sometimes it is also used as a measure of the area of non-grid drawings. It was easy to instruct our algorithm to prevent producing drawings with aspect ratio either higher than the one of the starting layout or higher than a given input value. What we simply had to do was to reject candidate positions, which violate this precondition.  

% ==================================================================
\myparagraph{Total resolution} 
% ==================================================================
%
The notion of the total resolution of a drawing was introduced relatively recently with aim of ``balancing'' the measures of the crossing and of the angular resolution of a drawing~\cite{DBLP:journals/cj/ArgyriouBS13}. Formally, it is defined as the minimum of these two measures. It was not difficult to adjust our algorithm to yield drawings with high total resolution by simply taking into account also the angular resolution of the drawing. In particular, if the total resolution of the drawing is defined by its angular resolution, then the way we compute the critical vertices of this drawing has to change; the critical vertices must be the endvertices of the pairs of edges that define the angular resolution. Also, at each iteration of our algorithm we have to ensure that the total resolution does not decrease. We do so by simply rejecting candidate positions, which yield a reduced total resolution.

% ==================================================================\myparagraph{Angular resolution} 
% ==================================================================
%  
As it is the case with the force-directed algorithms of Huang et al.~\cite{DBLP:journals/vlc/HuangEHL13} and Argyriou et al.~\cite{DBLP:journals/cj/ArgyriouBS13}, our algorithm can be also restricted to maximize only the angular resolution (by neglecting its crossing resolution). We already described in the previous paragraph the necessary changes in the definition of the critical vertices and the rule according to which a candidate position is rejected (i.e., when it yields a drawing with a reduced angular resolution).

% ==================================================================
\myparagraph{Grid drawings}
% ==================================================================
%
Our algorithm, as it has been described so far, does not necessarily produce grid drawings, i.e., drawings in which the vertices are at integer coordinates. However, it can be easily adjusted to produce such drawings. More precisely, if we round the candidate positions computed at each iteration of our algorithm to their closest grid points and use these grid points as candidates for the next position of the vertex to be moved, then the obtained drawing will be grid (assuming, of course, that the starting drawing is grid). One can even bound the size of the grid, which simply implies that candidate grid positions outside the bounds must be rejected. 

 %(the latter requirement is not difficult to be achieved, e.g., by slightly adjusting the positions of the vertices of a circular layout, whose underlying circle has large radius, to be at grid points). 

%An alternative approach is to let our algorithm compute a (non-grid) drawing and then apply some technique to convert it to grid. Critical here is not to affect the crossing resolution too much during the conversion. Of course, if the crossing resolution of the obtained grid drawing is at least as large as the crossing resolution of the non-grid starting drawing, then we considered the computed grid drawing \emph{acceptable}. In our approach, we also accepted drawings whose crossing resolution is at least as large as the crossing resolution of the non-grid starting drawing minus an input parameter $\epsilon$, which we used as \emph{slack} for tolerating slightly worse solutions. For converting a non-grid drawing to grid, we repeated the following procedure, that is also based on randomization, until we compute an acceptable grid drawing: For each vertex of the non-grid starting drawing, chose uniformly at random one of the grid points from the so-called \emph{neighbourhood} of the vertex, and move the vertex to this point (assuming of course that there is no other vertex at this point), where the neighbourhood of a vertex was initially set to its four closest grid points. If after a number of iterations an acceptable grid drawing without vertex-vertex overlaps could not be reported, then we augmented the neighbourhood of all vertices to contain more grid points, in particular, the grid points that are in horizontal and vertical distance~$1$ from the grid points of the current neighbourhood\todo{Not forget to mention that we may have to double the area.}. 


% ==================================================================
\section{Experimental Evaluation}
\label{sec:experiments}
% ==================================================================

In this section, we present the results of our experimental evaluation. For comparison purposes, apart from our algorithm, we have also implemented the force-directed algorithms of Argyriou et al.~\cite{DBLP:journals/cj/ArgyriouBS13} and Huang et al.~\cite{DBLP:journals/vlc/HuangEHL13}. The implementations were in Java using the yFiles library~\cite{DBLP:books/sp/04/WieseE004}. The experiment was performed on a Linux laptop with four cores at 2.4 GHz and 8 GB RAM.
%
As a test set for our experiment, we used the non-planar Rome graphs~\cite{DBLP:reference/crc/BattistaD13}, which form a collection of around 8.100 benchmark graphs (commonly used for testing the efficiency of algorithms for drawing graphs). 

The experiment was performed as follows. Initially, each Rome graph was laid out using the SmartOrganic layouter of yFiles~\cite{DBLP:books/sp/04/WieseE004}. Starting from this layout, every graph was drawn with 
\begin{inparaenum}[(i)]
\item our algorithm, 
\item our algorithm adjusted not to violate the aspect ratio of the initial layout, 
\item the force-directed algorithm of Argyriou et al.~\cite{DBLP:journals/cj/ArgyriouBS13}, and
\item the force-directed algorithm of Huang et al.~\cite{DBLP:journals/vlc/HuangEHL13}.
\end{inparaenum}
%
We compared the quality of the produced drawings based on the following aesthetic properties:
%
\begin{enumerate}[P.1.]
\item \label{p:cr-res} crossing resolution, 
\item \label{p:to-res} total resolution, 
\item \label{p:an-res} angular resolution, 
\item \label{p:as-rat} aspect ratio, and
\item \label{p:no-xing} number of crossings.
\end{enumerate}

The reported results are on average across different drawings with same number of vertices. For Properties~P.\ref*{p:cr-res}, P.\ref*{p:to-res} and P.\ref*{p:an-res}, we further adjusted each of the algorithms of the experiment to optimize exclusively the corresponding measures. Then, for each of them we reported the average of the corresponding aspect ratios and number of crossings of the produced layouts. Note that all algorithms of the experiment can easily be adjusted to maximize only the crossing resolution, or only the angular resolution or both (by maximizing the total resolution). In our algorithm, this can be achieved by adjusting appropriately the content of the vertex-pool (as we saw in Section~\ref{ssec:variants}), while in the algorithms of Argyriou et al.~\cite{DBLP:journals/cj/ArgyriouBS13} and of Huang et al.~\cite{DBLP:journals/vlc/HuangEHL13} by switching on only the forces that maximize the corresponding properties under measure (note that, each of these two algorithms has a different set of forces to maximize the crossing and the angular resolution, such that together they maximize the total resolution). 


\myparagraph{Crossing resolution} Our results for the crossing resolution are summarized in Fig.~\ref{fig:cr-res}. Here, each algorithm was adjusted to maximize exclusively the crossing resolution (i.e., by ignoring the drawing's angular resolution). It is immediate to see that our algorithm outperforms all other ones in terms of the crossing resolution of the produced drawings, when we do not impose any restriction on the aspect ratio of the computed drawings; refer to the solid-black curve, denoted as \emph{unrestricted}, in Fig.~\ref{fig:cr-res-1}. The variant of our algorithm, which does not violate the aspect ratio of the initial layout, leads to drawings with slightly smaller crossing resolution; refer to the solid-gray curve, denoted as \emph{ar-restricted}, in Fig.~\ref{fig:cr-res-1}. Finally, the two force-directed algorithms seem to produce drawings with worse crossing resolution; refer to the dotted-gray and dotted-black curves of Fig.~\ref{fig:cr-res-1} (by Argyriou et al.~\cite{DBLP:journals/cj/ArgyriouBS13} and by Huang et al.~\cite{DBLP:journals/vlc/HuangEHL13}, respectively). 

While our unrestricted algorithm produces drawings of better crossing resolution, this seems to come at a cost of drastically increased aspect ratio (see Fig.~\ref{fig:cr-res-2}), which, however, is still better that the corresponding aspect ratio of the drawings produced by the algorithm of Argyriou et al.~\cite{DBLP:journals/cj/ArgyriouBS13}. For the latter algorithm, it seems that the forces due to angles formed at the crossings outperform the corresponding spring forces, which try to keep the lengths of the edges short. Going back to our unrestricted algorithm, its behaviour is up to a certain degree expected mainly due to the fact that there is no control on the lengths of the edges.
% (as it is, e.g., in the case of the force-directed algorithm of Huang et al.~\cite{DBLP:journals/vlc/HuangEHL13}, whose sping forces inpose stong restrictions on the edge lengths). 
On the other hand, the restricted variant of our algorithm, which does not allow the aspect ratio to increase, has more or less comparable performace (in terms of the aspect ratio of the computed drawings) as the one of Huang et al.~\cite{DBLP:journals/vlc/HuangEHL13}.

Regarding the number of crossings, we observe that the restricted variant of our algorithm and the force-directed algorithm of Huang et al.~\cite{DBLP:journals/vlc/HuangEHL13} yield drawings with comparable number of crossings, which at the same time is significantly smaller than the corresponding number of crossings produced by the two other algorithms of our experiment; refer to Fig.~\ref{fig:cr-res-3}. 

A different behaviour can be observed in the number of iterations, which are required by the algorithms to converge; refer to Fig.~\ref{fig:cr-res-4}. We note here that we used different criteria to determine whether our algorithm (and its variant) and whether the two force-directed algorithms have converged, mainly due to the different nature of these algorithms. For each of the two force-directed algorithms, if the crossing resolution between two consecutive iterations was not improved more than 0.001 degrees, we assumed that the algorithm had converged and we did not proceed any more. On the other hand, for both of our algorithms we used a more relaxed criterion, because these algorithms may perform several iterations without changing the drawing (and thus, its crossing resolution), as they based on randomization. However, if after 100 consecutive iterations there was no improvement, then we assumed that the algorithm had converged and we did not proceed any more. The maximum number of iterations that each of the algorithms of our experiment could perform in order to converge was set to 100.000. By Fig.~\ref{fig:cr-res-4}, it is immediate to see that both the unrestricted and the restricted variant of our algorithm require less iterations than the two force-directed algorithms. 

\myparagraph{Total resolution} Our results for the total resolution are summarized in Fig.~\ref{fig:to-res}. Here, each algorithm was adjusted so to maximize the minimum of the crossing and of the angular resolution. We observe that both variants of our algorithm as well as the force-directed algorithm by Argyriou et al.~\cite{DBLP:journals/cj/ArgyriouBS13} tend to produce drawings of the same total resolution. Slightly worse is the performance of the algorithm by Huang et al.~\cite{DBLP:journals/vlc/HuangEHL13}; see Fig.~\ref{fig:to-res-1}. However, the aspect ratio of the drawings of the latter algorithm is better than the corresponding aspect ratios of the remaining algorithms of the experiment; see Fig.~\ref{fig:to-res-2}. More concritely, the drawings produced by the restricted variant of our algorithm have slightly worse aspect ratios. Then, the ones produced by the force directed algorithm of Argyriou et al.~\cite{DBLP:journals/cj/ArgyriouBS13} follow. However, we again observe that our unrestricted algorithm leads to drawings with very high aspect ratio. 

Similar observations can be made also for the number of crossings of the produced layouts; see Fig.~\ref{fig:to-res-3}. The algorithm by Huang et al.~\cite{DBLP:journals/vlc/HuangEHL13} yields drawings with the least number of crossings. Comparable but slightly worse (in terms of the number of crossings) are the drawings produced by the restricted variant of our algorithm and by the force-directed algorithm by Argyriou et al.~\cite{DBLP:journals/cj/ArgyriouBS13}. Our unrestricted algorithm seems to require the largest number of crossings, which turns out to be notably higher than the corresponding ones of the remaining algorithms of our experiment especially for large graphs. On the positive side, both the unrestricted and the restricted variant of our algorithm require less iterations than the two force-directed algorithms, as it can be easily seen in Fig.~\ref{fig:to-res-4}.

\myparagraph{Angular resolution} We conclude the analysis of our experimental evaluation with the  results for the angular resolution; refer to Fig.~\ref{fig:an-res}. Here, each algorithm was adjusted  to maximize exclusively the angular resolution (i.e., by ignoring the drawing's crossing resolution). A notable observation is that the force-directed algorithm by Argyriou et al.~\cite{DBLP:journals/cj/ArgyriouBS13} tends to have the same performance (in term of angular resolution) as our unrestricted algorithm, which is slightly better but mostly for small graphs; see Fig.~\ref{fig:an-res-1}. The restricted variant of our algorithm yields drawings with slightly worse angular resolution. The force-directed algorithm by Huang et al.~\cite{DBLP:journals/vlc/HuangEHL13} seems to be outperformed by all algorithms of the experiment. 

On the other hand, the results for the aspect ratio, the number of crossings and the required number of iterations are very similar with corresponding ones for the total resolution; see Figs.~\ref{fig:an-res-2}--\ref{fig:an-res-4}. This observation suggests that, for most of the graphs of our experiment, the angular resolution dominates the crossing resolution (and thus is the one defining the total resolution) in the constructed drawings, which explains the similarity in the reported results.  

% ==================================================================
\section{Conclusions}
\label{sec:conclusions}
% ==================================================================

\bibliographystyle{abbrvurl}
\bibliography{references}

\end{document}
